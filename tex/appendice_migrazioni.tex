\section{Migrazioni}

 \label{App:AppendiceMigrazioni}
Una migrazione è un file nel quale viene indicato cosa cambia nella struttura del database rispetto allo stato precedente ad essa e le variazioni ai dati ad essa conseguente.  Questo permette a tutti gli sviluppatori di avere lo stato della struttura del database sempre aggiornato.
Qui di seguito un esempio di migrazione che crea una tabella Bookmark (nel metodo statico up), e specifica cosa fare in caso di annullamento della migrazione (nel metodo statico down).
\begin{lstlisting}
class CreateBookmarks < ActiveRecord::Migration
  def self.up
    create_table :bookmarks do |t|
      t.string :url
      t.string :title
      t.text :description
      
      t.timestamps
    end
  end

  def self.down
    drop_table :bookmarks
  end
end
\end{lstlisting}
Una volta eseguita la migrazione verrà creata una tabella Bookmarks con i tre campi \textit{url}, \textit{title} e \textit{description}.