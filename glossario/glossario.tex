% per essere usato è necessario il pacchetto \usepackage{glossaries}
% da includere nel file in cui si deve usare tramite il comando \loadglsentries{glossario/glossary.tex}
\makeglossaries


% GLOSSARIO DEI TERMINI
%\appendix


\newglossaryentry{Angular}{name={Angular.js}, description={Framework javaScript open source per la realizzazione di applicazioni web. Si focalizza in particolare nella parte di front-end delle applicazioni ed utilizza un pattern architetturale di tipo MVC. La sua struttura permette di inserire delle keyword nelle view che andranno ad invocare delle direttive. Queste ultime inietteranno codice HTML al loro posto a seconda della logica gestita nei controller ed ai dati presenti nel model.}}

\newglossaryentry{Astah}{name={Astah}, description={Software proprietario usato per la modellazione di diagrammi UML.}}

\newglossaryentry{back-end}{name={Back-End}, description={\'E la componente software responsa di gestire le interazioni provenienti dall'esterno del sistema e restituire le risposte a seguito delle opportune elaborazioni. Si occupa inoltre di gestire la persistenza dei dati interagendo con eventuali basi di dati. Il back-end è responsabile dell'utilizzo dei dati forniti dal front-end.}}

\newglossaryentry{baseline}{name={Baseline}, description={Una baseline rappresenta uno stato di avanzamento consolidato all'interno del progetto.}}

\newglossaryentry{Bootstrap}{name={Bootstrap}, description={Framework basato su tecnologie HTML, CSS e JavaScript che fornisce delle classi css e delle funzionalità che permettono di sviluppare velocemente siti web responsive. Le sue caratteristiche permettono di concentrarsi sulla logica applicativa senza dedicare troppo tempo a curare l'aspetto grafico.}}

\newglossaryentry{bottom-up}{name={bottom-up}, description={Tecnica che impone di definire prima le parti individuali di un sistema che poi vengono connesse tra loro per formare componenti più complessi.}}

\newglossaryentry{branch}{name={Branch}, description={Ramo di un sistema di gestione per gestire i file separatamente.}}


\newglossaryentry{business}{name={Business}, description={Affare, commercio.}}

\newglossaryentry{casi d'uso}{name={Casi d'uso}, description={Tecnica usata nei processi di ingegneria del software per effettuare in maniera esaustiva e non ambigua, la raccolta dei requisiti. Essa consiste nel valutare ogni requisito focalizzandosi sugli attori che interagiscono col sistema, valutandone le varie interazioni.}}

\newglossaryentry{Chart.js}{name={Chart.js}, description={\'E una libreria JavaScript che supporta la generazione di grafici utilizzando i canvas di HTML5.}}

\newglossaryentry{conciso}{name={Conciso}, description={Modo di esprimersi breve, chiaro e denso di contenuto.}}

\newglossaryentry{Fabric.js}{name={Fabric.js}, description={Fabric.js è una libreria javascript che permette di collocare e modificare oggetti su un canvas. Mette a disposizione un strumenti per il parsing e la serializzazione del canvas in SVG e JSON}}

\newglossaryentry{database}{name={Database}, description={ Archivio di dati strutturato in modo da razionalizzare la gestione, l'immagazzinamento e l'aggiornamento delle informazioni e da permettere lo svolgimento di ricerche complesse.}}

\newglossaryentry{Design pattern}{name={Design pattern}, description={Si tratta di una descrizione o modello logico da applicare per la risoluzione di un problema ricorrente durante le fasi di progettazione e sviluppo del software.}}

\newglossaryentry{Facebook}{name={Facebook}, description={Servizio di rete sociale lanciato nel febbraio del 2004.}}

\newglossaryentry{feedback}{name={Feedback}, description={Processo per cui il risultato dell'azione di un sistema si riflette sul sistema stesso per correggerne o modificarne il comportamento}}

\newglossaryentry{framework}{name={framework}, description={Architettura software sulla quale un software può essere progettato e realizzato.}}

\newglossaryentry{front-end}{name={front-end}, description={Parte di un sistema software che gestisce l'interazione con l'utente o con sistemi esterni che producono dati di ingresso.}}

\newglossaryentry{font}{name={Font}, description={Insieme di caratteri tipografici caratterizzati e accomunati da un certo stile grafico o intesi per svolgere una data funzione.}}

\newglossaryentry{Gantt}{name={Gantt}, description={Il diagramma di Gantt è uno strumento di supporto alla gestione dei progetti che permette la rappresentazione grafica di un calendario di attività. Esso è utile al fine di pianificare, coordinare e tracciare specifiche attività in un progetto dandone una chiara illustrazione dello stato d'avanzamento.}}

\newglossaryentry{GanttProject}{name={GanttProject}, description={\'E un'applicazione desktop che fornisce supporto per il project management permettendo di realizzare i più importanti grafici per il controllo delle attivit\'a.}}

\newglossaryentry{Git}{name={Git}, description={\'E un software di controllo di versione distribuito, creato da Linus Torvalds.}}

\newglossaryentry{Github}{name={Github}, description={Servizio web di hosting per lo sviluppo di progetti software che usa il sistema di controllo di versione Git.}}

\newglossaryentry{Google Calendar}{name={Google Calendar}, description={Google Calendar è un sistema fornito gratuitamente da Google che permette di avere un calendario condiviso disponibile online nella quale inserire delle attvità ed impostare degli allarmi.}}

\newglossaryentry{Google Docs}{name={Google Docs}, description={Fa parte di una serie di applicazioni Web e di Office automation offerte da Google. Esso permette di salvare documenti di testo e fogli di calcolo nei formati .doc, .odt e .pdf, creare delle presentazioni, fogli di calcolo e moduli HTML.}}

\newglossaryentry{Google Drive}{name={Google Drive}, description={Servizio, in ambiente cloud computing, di memorizzazione e sincronizzazione online introdotto da Google che permette la condivisione e la modifica concorrente di documenti online.}}

\newglossaryentry{Google Hangout}{name={Google Hangout}, description={Software di messaggistica istantanea e VoIP sviluppato da Google.}}

\newglossaryentry{Google+}{name={Google+}, description={Rete sociale gratuita creata da Google che cerca di modellare il concetto di amicizia sotto forma di cerchie.}}

\newglossaryentry{hosting}{name={Hosting}, description={\'E un servizio che permette ad utenti ed organizzazioni di rendere accessibili da Internet i contenuti caricati su una macchina remota.}}



\newglossaryentry{HTML5}{name={HTML5}, description={Linguaggio di markup per la strutturazione delle pagine web, da ottobre 2014 pubblicato come W3C Recommendation}}

\newglossaryentry{indice Gulpease}{name={Indice Gulpease}, description={L'Indice Gulpease è un indice di leggibilità di un testo tarato sulla lingua italiana. Rispetto ad altri ha il vantaggio di utilizzare la lunghezza delle parole in lettere anziché in sillabe, semplificandone il calcolo automatico.}}

\newglossaryentry{infografica}{name={Infografica}, description={Tecnica che consiste nel rappresentare l'informazione in forma grafica per dare un'idea d'insieme sul contenuto da rappresentare.}}

\newglossaryentry{Inspection}{name={Inspection}, description={ Tecnica di esame che include misurazioni e test applicati su alcune caratteristiche proprie di un oggetto o di una attività.}}

\newglossaryentry{Java}{name={Java}, description={Linguaggio di programmazione orientato agli oggetti che permette di sviluppare programmi eseguibili su diverse tipologie di dispositivi e compatibili con qualsiasi sistema operativo che abbia installato un apposito ambiente virtuale chiamato JVM(Java Virtual Machine).}}

\newglossaryentry{Javascript}{name={JavaScript}, description={Un linguaggio di scripting spesso utilizzato in pagine web per la gestione di effetti dinamici interattivi tramite funzioni invocate da eventi attivati dall'utente oppure da altri eventi.}}

\newglossaryentry{jQuery}{name={JQuery}, description={jQuery è una libreria di funzioni Javascript per le applicazioni web. Essa si pone come obiettivo quello di semplificare la manipolazione, la gestione degli eventi e l'animazione delle pagine HTML.}}

\newglossaryentry{laravel}{name={Laravel}, description={Framework open source di tipo MVC scritto in PHP per lo sviluppo di applicazioni web.}}

\newglossaryentry{layout}{name={Layout}, description={\'E la struttura grafica di un sito web o di un documento.}}

\newglossaryentry{Linguaggio di markup}{name={Linguaggio di markup}, description={Un linguaggio di markup è un insieme di regole che descrivono i meccanismi di rappresentazione (strutturali, semantici o presentazionali) di un testo che, utilizzando convenzioni standardizzate, sono utilizzabili su più supporti.}}

\newglossaryentry{Linguaggio di programmazione}{name={Linguaggio di programmazione}, description={Insieme di regole che descrivono i meccanismi di rappresentazione di un testo che, utilizzando convenzioni standardizzate, sono utilizzabili su più supporti. La tecnica di composizione di un testo con l'uso di marcatori (o espressioni codificate), richiede quindi una serie di convenzioni, ovvero di un linguaggio a marcatori di documenti.}}

\newglossaryentry{Linux}{name={Linux}, description={Famiglia di sistemi operativi di tipo Unix-like, rilasciati sotto varie distribuzioni, aventi la caratteristica comune di utilizzare come nucleo il kernel Linux}}

\newglossaryentry{Mac OSX}{name={Mac Os}, description={Sistema operativo di Apple dedicato ai computer Macintosh. Il nome è l'acronimo di Macintosh Operating System.}}

\newglossaryentry{Mercurial}{name={Mercurial}, description={Mercurial è un software multipiattaforma di controllo di versione distribuito creato da Matt Mackall e rilasciato sotto GNU General Public License 2.0.}}

\newglossaryentry{Middleware}{name={Middleware}, description={Insieme di programmi informatici che fungono da intermediari tra diverse applicazioni e componenti software; Sono spesso utilizzati come supporto per sistemi distribuiti complessi}}

\newglossaryentry{milestone}{name={Milestone}, description={Termine inglese che letteralmente significa pietra miliare. Viene tipicamente utilizzato nella pianificazione e gestione di progetti complessi per indicare il raggiungimento di obiettivi stabiliti in fase di definizione del progetto stesso. Le milestone indicano importanti traguardi intermedi nello svolgimento del progetto}}

\newglossaryentry{MongoDB}{name={MongoDB}, description={MongoDB (da "huMONGOus", enorme) è un DBMS non relazionale, orientato ai documenti. Classificato come un database di tipo NoSQL, MongoDB preferisce alla struttura tradizionale basata su tabelle dei database relazionali una struttura basata su documenti in stile JSON, rendendo l'integrazione di dati di alcuni tipi di applicazioni più facile e veloce.}}

\newglossaryentry{Mustache}{name={Mustache}, description={Sistema che si occupa di realizzare dei template che sostituiscono a sezioni di codice complesse un segnaposto; esso verrà poi sostituito con il codice effettivo solo quando è necessario. Esso è disponibile per diversi linguaggi di programmazione.}}

\newglossaryentry{parser}{name={Parser}, description={Un parser è un programma che analizza un flusso continuo di dati in ingresso in modo da determinare la sua struttura grazie ad una data grammatica formale.}}

\newglossaryentry{PDCA}{name={PDCA}, description={(Plan-Do-Check-Act) è un metodo di gestione in quattro fasi iterativo utilizzato in attività per il controllo e il miglioramento continuo dei processi e dei prodotti. \'E noto anche come il \textit{ciclo di Deming}.}}

\newglossaryentry{php}{name={Php}, description={Linguaggio di programmazione interpretato, originariamente concepito per la programmazione di pagine web dinamiche, principalmente utilizzato per sviluppare applicazioni web lato server, ma può essere usato anche per scrivere script a riga di comando o applicazioni stand-alone con interfaccia grafica.}}

\newglossaryentry{plugin}{name={Plugin}, description={Programma non autonomo che interagisce con un altro programma per ampliarne o estenderne le funzionalità originarie.}}

\newglossaryentry{prototipo}{name={Prototipo}, description={Un prototipo è un modello originale o il primo esemplare di un prodotto, rispetto a una sequenza di eguali o similari realizzazioni successive.}}

\newglossaryentry{real time}{name={Real time}, description={Real-time (in italiano "tempo reale") è un termine utilizzato in ambito informatico per indicare quei programmi che ottengono o richiedono informazioni aggiornate all'istante della richiesta.}}

\newglossaryentry{repository}{name={Repository}, description={Database in grado di contenere svariate tipologie di dati, corredate da relative informazioni (metadati). Offre inoltre un sistema di versionamento in grado di tener traccia delle modifiche effettuate al suo interno.}}


\newglossaryentry{REST-like}{name={REST-like}, description={Tipo di architettura REST ma con regole meno stringenti.}}

\newglossaryentry{Reveal.js}{name={Reveal.js}, description={Framework per presentazioni di slide.}}

\newglossaryentry{Same Origin Policy}{name={Same Origin Policy}, description={Regola che impedisce ad uno script di accedere a metodi e proprietà di pagine provenienti da siti diversi da quello su cui risiede.}}

\newglossaryentry{Servizio cloud}{name={Servizio cloud}, description={Servizio che fornisce la possibilità di erogazione di risorse informatiche, come l'archiviazione, l'elaborazione o la trasmissione di daticaricati e/o già presenti in rete.}}

\newglossaryentry{slack}{name={Slack}, description={Periodo di inatività.}}

\newglossaryentry{slide}{name={Slide}, description={Diapositiva digitale.}}

\newglossaryentry{sourceforge}{name={Sourceforge}, description={\'E una piattaforma e un sito web che fornisce gli strumenti per portare avanti un progetto di sviluppo software in modo collaborativo tra più sviluppatori.}}

\newglossaryentry{Task}{name={Task}, description={\'E un'attività che deve essere realizzata entro un determinato periodo di tempo.}}

\newglossaryentry{Task list}{name={Task list}, description={Una lista di task.}}

\newglossaryentry{template}{name={Template}, description={Struttura generica o standard che fornisce un modello comune su cui inserire informazioni diverse.}}

\newglossaryentry{ticket}{name={Ticket}, description={Rischiesta di esecuzione relativa ad uno specifico compito assegnata ad un numero ristretto di persone.}}

\newglossaryentry{Ticketing}{name={Ticketing}, description={Sistema con cui si richiede a qualcuno di eseguire un compito.}}

\newglossaryentry{top-down}{name={Top-Down}, description={Tecnica di progettazione software nel quale si formula inizialmente una visione generale del sistema per poi successivamente aggiungere maggiori dettagli di progettazione. Ogni nuova parte così ottenuta può quindi essere nuovamente rifinita, specificando ulteriori dettagli, finché la specifica completa è sufficientemente dettagliata da validare il modello.}}

\newglossaryentry{Trait}{name={Trait}, description={I Trait sono un meccanismo per il riuso del codice in linguaggi ad ereditarietà singola, come PHP. L'obiettivo di un Trait è quello di ridurre alcune delle limitazioni dovute a questa tipologia di ereditarietà, permettendo di definire un determinato set di metodi da riusare in più classi che non sono necessariamente nella stessa gerarchia}}

\newglossaryentry{Twitter}{name={Twitter}, description={Servizio gratuito di social networking e microblogging che fornisce agli utenti una pagina personale aggiornabile tramite messaggi di testo con una lunghezza massima di 140 caratteri}}

\newglossaryentry{Ubuntu}{name={Ubuntu}, description={Distribuzione GNU/Linux, basata su Debian.}}

\newglossaryentry{verbale}{name={Verbale}, description={Documento che contiene una verbalizzazione consistente nella narrazione per iscritto, in maniera sintetica ma fedele, fatta dalla persona incaricata, di dichiarazioni, operazioni o altri fatti avvenuti in sua presenza, allo scopo di ricordarli e costituirne prova.}}

\newglossaryentry{Versionamento}{name={Versionamento}, description={Gestione di versioni multiple di documenti o insiemi di informazioni.}}

\newglossaryentry{Walkthrough}{name={Walkthrough}, description={Tecnica di esaminazione nel quale si ispeziona meticolosamente un documento per ricercarne eventuali errori.}}

\newglossaryentry{web-scraping}{name={Web-Scraping}, description={Tecnica informatica di estrazione di dati da un sito web mediante l'utilizzo di programmi software.}}

\newglossaryentry{Skype}{name={Skype}, description={Software proprietario gratuito di messaggistica istantanea e VoIP.}}

\newglossaryentry{Windows}{name={Windows}, description={Famiglia di ambienti operativi e sistemi operativi dedicati ai personal computer, alle workstation, ai server e agli smartphone.}}

\newglossaryentry{WordPress}{name={WordPress}, description={\'{E} una piattaforma software di "personal publishing" e content management system (CMS), sviluppata in PHP appoggiandosi al gestore di database MySQL. Essa consente la creazione e distribuzione di un sito Internet formato da contenuti testuali o multimediali.}}

%\newglossaryentry{}{name={}, description={}}

% GLOSSARIO DEGLI ACRONIMI

\newacronym{CMS}{CMS}{Content Management System}
\newacronym{CRUD}{CRUD}{Create, Read, Update, Delete}
\newacronym{CSS}{CSS}{Cascading Style Sheets}
\newacronym{GUI}{GUI}{Graphical User Interface}
\newacronym{JSON}{JSON}{JavaScript Object Notation}
\newacronym{ORM}{ORM}{Object Relational Mapping}
\newacronym{PERT}{PERT}{Project Evaluation and Review Technique}
\newacronym{PNG}{PNG}{Portable Network Graphics}
\newacronym{SDK}{SDK}{Software Development Kit}
\newacronym{SEMAT}{SEMAT}{Software Engineering Method and Theory}
\newacronym{SVN}{SVN}{Subversion}
\newacronym{SWE}{SWE}{Software Engineering}
\newacronym{UI}{UI}{User Interface}
\newacronym{UML}{UML}{Unified Modeling Language}
\newacronym{URI}{URI}{Uniform Resource Identifier}
\newacronym{WBS}{WBS}{Work Breakdown Structure}
\newacronym{W3C}{W3C}{World Wide Web Consortium}
%\newacronym{}{}{}






\newacronym{AWS}{AWS}{Amazon Web Services}
\newacronym{DOM}{DOM}{Document Object Model}
\newacronym{DPI}{DPI}{Dispositivi di Protezione Individuale}
\newacronym{DPC}{DPC}{Dispositivi di Protezione Collettivi}
\newacronym{DVR}{DVR}{Documento di Valutazione dei Rischi}
\newacronym{INAIL}{INAIL}{Istituto Nazionale per l'Assicurazione contro gli Infortuni sul Lavoro}
\newacronym{ISTAT}{ISTAT}{Istituto Nazionale di Statistica}
\newacronym{JVM}{JVM}{Java Virtual Machine}
\newacronym{MVC}{MVC}{Model View Controller}
\newacronym{SaaS}{SaaS}{Software as a Service}

\newglossaryentry{HTML}{name={HTML}, description={HyperText Markup Language. Linguaggio di markup per la strutturazione delle pagine web.}}
\newglossaryentry{ATECO}{name={ATECO},description={ Sigla della classificazione ISTAT(Istituto Nazionale di Statistica) per determinare la tipologia delle attività economiche (\textbf{AT}tività \textbf{ECO}nomiche).}}
\newglossaryentry{reflection}{name={Reflection},description={Capacità di un programma di eseguire elaborazioni che hanno per oggetto il programma stesso, ed in particolare la struttura del suo codice sorgente. Un utilizzo comune, è l'utilizzo di questo approccio per individuare il nome della classe oppure degli attributi a partire da una istanza di un oggetto.}}
\newglossaryentry{SASS}{name={SASS},description={È un preprocessore CSS che analizza il codice scritto in SCSS e ne elabora il risultato. Si tratta a tutti gli effetti di una estensione di CSS3 che aggiunge:
	\begin{itemize}
		\item Regole annidate;
		\item Variabili;
		\item Mixin;
		\item Ereditarietà tra i selettori.
	\end{itemize}
Home page del progetto: \url{http://sass-lang.com/}.
	}}
	
	
	\newglossaryentry{REST}{name={REST}, description={Tipo di architettura software per i sistemi di ipertesto distribuiti come il World Wide Web, si riferisce ad un insieme di principi di architetture di rete, i quali delineano come le risorse sono definite e indirizzate.}}
	
\newglossaryentry{SCSS}{name={SCSS},description={Sta per Sassy CSS ed è la sintassi utilizzata da SASS. }}
\newglossaryentry{CDN}{name={CDN},description={Sta per\textit{ Content Delivery Network}, ovvero rete di consegna di contenuti. Questo meccanismo viene utilizzato per reperire il codice relativo agli strumenti di terze parti direttamente dalla rete, senza tenere una versione del codice stesso nel proprio server web.}}
\newglossaryentry{Breadcrumb}{name={Breadcrumb},description={Letteralmente si traduce in \textit{briciole di pane}. Rappresenta il percorso dalla home page, o comunque da una radice, al punto corrente della navigazione. \\ Il nome deriva dalla fiaba di Hansel e Gretel, in cui Hansel sparge lungo il suo cammino delle briciole di pane per poter riconoscere la strada precedentemente percorsa e poter tornare sui suoi passi. }}

\newglossaryentry{javascript}{name={Javascript},description={TODO}}
\newglossaryentry{asseverazione}{name={Asseverazione},description={Con asseverazione si intende una scelta volontaria di una impresa edile al fine di dimostrare l'impegno per la prevenzione, salute e sicurezza nei luoghi di lavoro. Opera mediante la certificazione di conformità dei modelli di organizzazione e gestione aziendali e mediante visite a campione nei cantieri. L'asseverazione offre benefici economici alle aziende che la richiedono e garantisce efficacia esimente rispetto alla responsabilità amministrativa delle imprese.}}
\newglossaryentry{concern}{name={Concern},description={Un concern rappresenta un modulo di Rails utilizzato per facilitare la gestione di funzionalità assegnabili a più classi ma allo stesso tempo gestibili in modo atomico. }}
\newglossaryentry{Ajax}{name={Ajax}, description={Tecnica di sviluppo software per la realizzazione di applicazioni web interattive. Lo sviluppo di applicazioni HTML con AJAX si basa su uno scambio di dati in background fra web browser e server, che consente l'aggiornamento dinamico di una pagina web senza esplicito ricaricamento da parte dell'utente.}}

\newglossaryentry{browser}{name={browser}, description={Un browser è un programma che consente di visualizzare i contenuti delle pagine web e di interagire con esse.}}

\newglossaryentry{reverse proxy}{name={Reverse proxy}, description={Si tratta di un tipo particolare di proxy che recupera le informazioni per conto di un client da uno o più server. Le risorse ottenute vengono restituite al client come se provenissero direttamente dal Proxy server, rendendo trasparente il recupero distribuito delle informazioni all'utente. Questo tipo di approccio risulta molto utile nei contesti in cui il recupero delle informazioni risulta oneroso perché, permettendo la distribuzione su più macchine, si ottiene un bilanciamento del carico.}}

\newglossaryentry{POJO}{name={POJO},description={Sta per \texttt{Plain Old Java Object}. Rappresenta un oggetto Java non legato ad alcuna restrizione diversa da quelle costrette dalla specifica del linguaggio Java.}}

\newglossaryentry{JavaBean}{name={JavaBean},description={Letteralmente si traduce in "Chicchi di Java". Le JavaBean sono classi scritte in Java seguendo le seguenti convenzioni:
		\begin{itemize}
			\item Un costruttore senza parametri;
			\item Devono essere definiti tutti i metodi accessori per le variabili manipolabili (in particolare getter e setter);
			\item La classe dovrebbe essere serializzabile;
			\item La classe non dovrebbe contenere metodi per la gestione degli eventi.
		\end{itemize}
		Le classi JavaBean, sono spesso utilizzate per incapsulare più oggetti in un singolo oggetto in modo da passare un solo oggetto ai metodi che necessitano di tutti qusti oggetti come parametri.
		}}
	

\newglossaryentry{ERB}{name={ERB},description={Sta per Embedded RuBy. \\
		È il sistema per la gestione dei template in Rails. ERB permette di generare file in moltissimi formati generando a tutti gli effetti dei template. In particolare, le informazioni vengono reperite mediante Ruby On Rails, ma la struttura del file nel formato specificato rimane invariata, adattando a tutti gli effetti le informazioni presenti al mommento della invocazione ad una struttura definita in precedenza.}}

\newglossaryentry{LHS}{name={LHS},description={Sta per Left Hand Side. Rappresenta l'insieme delle condizioni che verificano una regola del rule engine Drools.}}

\newglossaryentry{RHS}{name={RHS},description={Sta per Right Hand Side. Rappresenta l'insieme delle azioni da intraprendere se tutte le condizioni presenti nella LHS del rule engine Drools sono soddisfatte. }}




